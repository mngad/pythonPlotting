\chapter{Bovine Tail Vertebrae Study}\label{chap_bov}

\section{Introduction}

This chapter is split into two main sections discussing initially the
development of experimental methods testing bovine tail vertebrae with the
second section describing the methods of computationally modelling these
vertebrae using FEA@. These main sections are split further into method
development, sensitivity tests and the results. In this chapter the
experimental and computational work is limited to bovine tail vertebrae due to
their plentiful nature and relatively similar geometry to human vertebrae,
while not having many of the problems with increased yield strength and density
of porcine or other tissue \cite{zapata2017methodology}.  Allowing translation
into using the same or similar methods with human lumbar vertebrae in the
following chapter.

\section{Experimental Methods}\label{experimental-methods}


\subsection{Introduction}

The experimental methods that have been developed and the early results
acquired in this sections allow easier transition to using human tissue  and
provide valuable results for the development of specimen specific finite
element models. Studying these bovine tail vertebrae allows the development of
methods for material testing, acquiring $\mu$CT scans of the specimens and
carrying out vertebroplasty on the specimens. This is in addition to developing
computational models of the vertebrae discussed in
\cref{finite-element-modelling-methods}. The following section will detail the
development of various aspects of the experimental procedure, difficulties
encountered and traversed, experimental results and finally a discussion of the
methods, results and future work.

The steps involved in the developed methods involve dissection of the soft
tissue from the vertebrae, potting in PMMA end-caps, scanning using a $\mu$CT
scanner, compression testing and augmentation, the order of which can be seen
in \cref{fig:exp_flowchart}.  Specimen preparation, fracture generation and
initial $\mu$CT scanning was undertaken jointly with Ruth Coe (PhD student,
University of Leeds). Vertebroplasty (following initial training attempts
carried out with Dr Peter Loughenbury \& Dr Vishal Borse from the Leeds General
Infirmary) and subsequent loading and scanning was carried out solely by the
author.

\begin{figure}[ht!]
\centering
\includegraphics[width=2in]{images/Exp_FlowChart.png}
\caption{Flow-chart detailing the experimental process from initial dissection to final load test.}
\label{fig:exp_flowchart}
\end{figure}

\subsection{Specimen Preparation}\label{specimen-preparation-bov}

Bovine tails were acquired from a local abattoir and frozen to -20$^\circ$C
prior to use.  They were defrosted in a 4\(^\circ\)C fridge for approximately
24 hours before the initial dissection. The three most caudal vertebral (CC1 to
CC3) were kept, discarding the remainder of the tail due to the elongation of
the vertebral body further distal of the first three vertebrae. In addition to
the elongation of the vertebral body the spinal canal narrows limiting its
ability to house a steel rod used for mounting the vertebrae in PMMA end-caps.
Soft tissue was removed from the vertebrae as thoroughly as possible, including
the intervertebral disc material and material occupying the spinal canal. This
was carried out in order to remove potential error when comparing experimental
results of stiffness to the vertebra models developed from $\mu$CT scans (due
to difficulties modelling the soft tissues) and to allow a metal rod through
the spinal canal to aid alignment.


\begin{figure}[ht!] \centering
\includegraphics[width=5in]{images/potting_vertebra.png} \caption{Photograph
and diagram depicting the method of creating end-caps for the specimens.}
\label{fig:potting_vertebra} \end{figure}



Once dissected (and in subsequent breaks between procedure steps) the vertebrae
were wrapped in phosphate buffered solution (PBS) soaked tissue, to limit the
drying of the vertebral bone. The specimens were potted in PMMA end-caps to
allow repeated loading of the vertebrae with the same orientation and
positioning, while constraining the vertebrae as little as possible and
allowing flexion of the upper endplate.  Such flexion (anterior and posterior
bending), occurs naturally in the human spine, hence representing this
experimentally is important.  The setup for potting the vertebrae can be seen
in \cref{fig:potting_vertebra}.  Vertebrae were held using retort stands and
clamps holding a rod placed through the spinal canal. Depending on the level of
the vertebrae the spinal canal was packed with foam around the rod forming a
snug fit while the vertebrae was held approximately 5 mm above a petroleum
jelly lubricated metal surface. Also, depending on the level of the vertebrae,
any pedicles that protruded past the limits of the metal cylinders were removed
with a hacksaw at their base to prevent issues with the loading and scanning
tests which followed, most often this was limited to the most caudal vertebrae.
The removed pedicles can be seen in \cref{fig:potting_vertebra}. Lubricated
hollow metal cylinders of \(\sim\)10 cm diameter were used to form the endplate
when the 2:1 powder to liquid component PMMA mixture was added. PMMA was added
until the endplate of the vertebral body was covered up to the point where the
body starts to become concave. After approximately 20 minutes the PMMA had
sufficiently set to turn over the vertebra and create the end-cap at the other
end using the same process with the addition of a level to ensure the creation
of parallel end-caps.

Once the PMMA was set the vertebrae were wrapped in more PBS soaked tissue
before being frozen or stored in a fridge until the vertebrae were loaded to
fracture. The specimens were frozen only if more than 24 hours would pass
before the next stage of testing to reduce the number of freeze thaw cycles.
\subsection{Axial Compression}\label{axial-compression}

\subsubsection{Fracture Creation}\label{fracture-creation}

All specimens underwent axial compression using a material testing machine in
order to generate fractures within the vertebral body. Mounted vertebrae were
placed between two steel end-plates, the lower of which contains four screws to
inhibit lateral motion of the specimen when under load and the upper plate
contains similar screws, with the addition of a chamfered hole. This chamfered
hole allows the alignment of the specimen so that the loading point was
directly below the head of the testing machine using the marker located above
the centre of the vertebral body. The steel ball becomes the centre of rotation
for the free to rotate upper end-cap. This permitted rotation mimics natural
loading of the vertebrae and increases the likelihood of physiological anterior
wedge fractures. Details of the setup can be seen in \cref{fig:expsetup}.

\begin{figure}[ht!]

\centering \includegraphics[width=3.68472in]{images/instrom_loading_diag.png}
\caption{The experimental setup for axial loading the vertebral specimens.}
\label{fig:expsetup} \end{figure}

Loading of the vertebrae starts with a preload from 50 N to 300 N for 10 cycles
at a rate of 1mm/minute to remove any viscoelastic effects of any remaining
soft tissue. Following the preload, displacement was increased by 1 mm/minute
until either the load reached 9500 N (a safety limit due to the 10 kN load cell
limit) or a visible failure occurred on the real-time load-displacement plot
during compression.  This failure was observed as a peak in load with the
compression being stopped once clear decrease in load was observed. Both
scenarios can be seen in \cref{fig:failure_non_failure}.



\begin{figure}[ht!]
\centering
\includegraphics[width=16cm]{images/comparing_failureand_nonfailure.png}
\caption{The difference between failure (A) and non-failure (B) for bovine tail vertebra compressed to a maximum load of 9500 N or until a peak was observed.}
\label{fig:failure_non_failure}
\end{figure}

\subsubsection{Post Fracture \& Post Augmentation Loading \& Stiffness
Calculation}\label{post-fracture-post-vertebroplasty}

In order to find the stiffness of the previously fractured and augmented
specimens a similar loading procedure was used. However, following the preload,
compression was stopped when the load reached 5000 N as a means to limit
additional damage and fractures to the vertebrae. This ensured that the
vertebral stiffness across the three stages (intact, post-fracture and
post-augmentation) was calculated from the same range of loads (0 - 5000 N). To
examine the effect that the initial load to failure has on the following loads,
both post-fracture and post-augmentation, a control specimen was used. This
control (T1 CC3) was only loaded up to 5000 N before ending the test.




%\begin{table}[ht!]
%\centering
%\caption{My caption}
%\label{my-label}
%\begin{tabular}{l|c|c|c}
%         & \multicolumn{3}{c}{Greatest Gradient (N/mm)}          \\ \hline
%Vertebra & \textless 1500 N & \textless 5000 N & \textless 9500 N \\ \hline \hline
%T8 CC1   & 4396.6           & 5914.3           & 6474.7           \\
%T8 CC2   & 7824.0           & 8132.8           & 8378.5           \\
%T8 CC3   & 3526.3           & 5680.6           & 6416.3           \\
%T9 CC1   & 4688.8           & 5833.5           & 6843.1           \\
%T9 CC2   & 3580.2           & 4449.8           & 4950.4           \\
%T9 CC3   & 3539.1           & 5093.7           & 5968.4
%\end{tabular}
%\end{table}

%add link to graph of load-disp showing wonky "linear region"

The stiffness of the specimens throughout their tests was calculated using a
Python script on the raw data from the materials testing machine.
The script allowed the limits of the range of interest to be set and,
using a defined segment size, incremented over the data reporting the greatest
stiffness found in a segment. The segment size was set to 0.3 mm. The script
iterated over the data in overlapping increments of 0.1 mm and the range of
interest was set to 0 - 5000 N, an explanation of this can be seen in \cref{fig:load_disp_incr}.

\begin{figure}[ht!]

\centering
\includegraphics[width=4.18472in]{images/load_disp_incr.png}
\caption{A typical load displacement curve showing how the gradient was taken from 0.3 mm long sections incremented at 0.1 mm across the length of the curve.}
\label{fig:load_disp_incr}
\end{figure}

If the load-displacement curves were perfectly linear within the ``linear
region" the stiffness in the three ranges of interest in
\cref{fig:barchartcompgrads} (0-1500 N, 0-5000 N and 0-9500 N) would give an equal value for the stiffness or
maximum gradient. However, given that these values were found to differ for the
three ranges it suggested non-linear behaviour.  As shown in
\cref{fig:barchartcompgrads} the recorded maximum stiffness varies greatly
depending on what portion of the load displacement graph was being examined.
With the maximum stiffness only measurable with the 0-9500 N range.
Given that the measured stiffness in the 0 to 5000 N is much closer to the
measurement with the full range is included, compared to only including the 0
to 1500 N, it was decided this was a reasonable trade off given the attempt to
limit the damage with post fracture tests and the need for a uniform
measurement range.

\begin{figure}[ht!]

\centering
\includegraphics[width=4.18472in]{images/barchartCompGrads.png}
\caption{The difference seen when measuring the greatest gradient (stiffness) using different portions of the load displacement curve. From 0 to 1500 N, 0 to 5000N and 0 to 9500N.}
\label{fig:barchartcompgrads}
\end{figure}


\subsection{Vertebroplasty }\label{vertebroplasty-bov} The procedure was
developed in collaboration with two clinicians (Dr Peter Loughenbury \& Dr
Vishal Borse) from the Leeds General Infirmary. Subsequent tests of the
procedure were undertaken with the aid of Dr Sebastien Sikora, Dr Fernando
Zapata Cornelio \& Ruth Coe (Research Fellow, Research Fellow \& PhD student
respectively), while the specimens presented in the results below were
augmented soley by the author.

Due to the differences between human and bovine
vertebrae it was not possible to perform bi-pedicular vertebroplasty on the
bovine specimens using the methodologies established for human vertebra.  The
main difficulty was the greatly increased density of the bovine vertebra bone,
meaning that rather than pushing the vertebroplasty needle into the vertebra by
hand, a mallet and vice to hold the vertebra were required. In addition to
this, the force required to inject cement into the vertebral body was greatly
increased.  The vertebroplasty method for bovine tail vertebra was therefore
developed over several iterations due to these difficulties. This sub-section
details the initial procedure, the problems encountered and solutions developed
to allow a clinically relevant volume of cement to be injected and captured in
$\mu$CT scans.

\subsubsection{Initial Procedure}

The procedure was initiated by using bone nibblers to remove the rounded end of
both posterior pedicles, providing a surface to start the needle entry. While
holding the vertebra in a table mounted vice the needle's 1 cm markings were used to
estimate the depth and angle needed to reach the anterior quarter of the
vertebral body. The placement of the needle required care to ensure the pedicle
was not damaged through splitting as it was inserted. A mallet was used to
insert the needle until it was at the depth required; the procedure was
repeated for the other pedicle, reusing the same needle.



The PMMA cement was mixed 1:1 monomer to powder to ensure that it could be
drawn up via the syringe and to allow enough time to inject the cement before
it thickened and set. This additional setting time and reduced viscosity is
also used by clinicians, who use ratios up to 0.74 monomer to powder with no
adverse outcomes associated despite the reduced modulus and strength often
reported \cite{Belkoff2002,Jasper1999}.  While the vertebra was held in the
clamp of a retort stand, the syringe was attached to the needle, which in turn
was inserted into one of the pre-made tracks through the pedicle into the
vertebral body.
%The syringe was then attached to the needle, with the interior rod removed,
%which was inserted to one side of the vertebra mounted using a retort stand.
Cement was pushed into the vertebrae using the syringe, until 3-4 mL was
inserted into both sides of the vertebrae, with cement being used to back fill
as the needle was removed to fill the channel created by the needle. The
vertebrae were then left for approximately an hour until the cement had set
before scanning.

% Add picture and clarify


\subsubsection{Complications and Changes to the Procedure}\label{complications}

Various problems were encountered while carrying out the procedure that
required the methods to be adapted. These challenges and their solutions are
described below.

\textbf{Vertebral Temperature:} The first of these was the difficulty found
injecting any cement into the vertebra. With the initial specimens, cement was
injected but it was mainly limited to the needle tracks rather than the
vertebral body. To counter this the vertebrae were warmed to 37$^\circ$C for an
hour or until the internal temperature of the vertebrae had reached this
temperature (using a temperature probe in the vertebroplasty needle hole). This
meant that the bone marrow inside the vertebrae was no longer solid and
therefore could be displaced by the cement making the injection much easier.

\textbf{Radio-opacity of Cement:} A second problem was the opacity of the
cement on $\mu$CT scans, which proved difficult to segment and separate it from
the trabeculae in the vertebral body as can be seen in
\cref{fig:withWithoutBaSO4}:A. Here, the cement was indistinguishable from the
bone marrow and can only be seen in the needle channel. The solution to this
was to mix barium sulphate (BaSO$_4$) with the PMMA to achieve the radio
opacity seen in \cref{fig:withWithoutBaSO4}:B, where the bright area in the
centre of the vertebral body is the injected cement and BaSO$_4$ combined. Due
to the hydrophilic nature of the BaSO$_4$ powder it was important to use a
completely dry beaker when thoroughly mixing it with the PMMA powder to limit
aggregation of the BaSO$_4$, which can be seen in the bright spots in
\cref{fig:withWithoutBaSO4}:B. The two components were used in a 1:4 BaSO$_4$
to PMMA powder ratio, mixed 1:1 with the liquid PMMA component.

\begin{figure}[ht!]
\centering
\includegraphics[width=4.18472in]{images/withWithoutBaSo4.png}
\caption{A: $\mu$CT scan of an augmented vertebrae, with some visible PMMA residing in the needle channel. B: $\mu$CT scan of an augmented vertebrae using PMMA mixed with barium sulphate.}
\label{fig:withWithoutBaSO4}
\end{figure}
\textbf{Cement Leaking from Vascular Channels:} Preventing the cement from
exiting the vertebrae from vascular channels while injecting the cement proved
to be another obstacle to achieving a physiologic fill volume for the
vertebrae. These channels lead both out the anterior face and from the
vertebral body into the spinal canal, this can be seen in
\cref{fig:cementleakage} and \ref{fig:2vertWithoutBluTac}. In the body these
channels would be filled with vasculature preventing the cement leaking through
them. Two main methods were used to stop cement leaking while carrying out the
procedure on the bovine tail vertebra. The first was to use the same rod used
for mounting the vertebrae in their end-caps to limit the passage of cement
into the spinal canal. The second was to use blu-tac to cover the external
vascular channels, wrapped with cling-film to hold it in place. This allowed
any bone marrow free passage out of the vertebrae, but enough resistance to
limit the flow of cement.

\begin{figure}[ht!]
\centering
\includegraphics[width=4.18472in]{images/cementleakage.png}
\caption{A: $\mu$CT scan of an augmented vertebrae showing the cement leaking from vascular channels on the anterior side. B: Photograph of an augmented vertebrae cut into four quarters showing a vascular channel leading into the spinal canal.}
\label{fig:cementleakage}
\end{figure}

\begin{figure}[ht!]
\centering
\includegraphics[width=4.18472in]{images/bigLeakage_noBluTac.png}
\caption{$\mu$CT scans of two vertebra, showing the cement leaking into the spinal canal and out of the vascular channels and the vertebral surface.}
\label{fig:2vertWithoutBluTac}
\end{figure}
% add figure of scan to make clear what the right hand pic is


\subsection{MicroCT Scanning}

$\mu$CT scans were taken at three occasions during the experimental process.
These scans occur before and after the initial load to failure, then following
the augmentation of the specimens. The process requires the vertebrae to be
defrosted and at room temperature, given that the radio-opacity of water
differs between solid and liquid states, hence vertebrae were usually defrosted
overnight in a 4$^\circ$C fridge. Vertebrae were loaded two at a time in a
carbon fibre loading cradle into a HR-pQCT (XtremeCT, Scanco Medical
AG, Switzerland) scanner. The settings used for the scans were: an isotropic
voxel size of 82 $\mu$m, energy settings 900 $\mu$A, 60 kVp and 300 ms exposure
time. These settings were based upon previous studies using the same scanner
and similar vertebrae carried out in the group \cite{zapata2017methodology,
Sikora2013a, RobsonBrown2014, Wijayathunga2008}.

\subsection{Results}\label{results}

The stiffness values for 12 vertebra from 6 bovine tails are shown in \cref{fig:allexpData}. Of the twelve vertebra only two,
the first and second tail vertebra of the second tail (T2 CC1 \& T2 CC2), were
fractured. The remaining nine (excluding T1 CC3, the control) reached 9500 N
and therefore did not fail.

\begin{figure}[ht!]
\centering
\includegraphics[width=\textwidth]{images/All_experimental_Data.png}
\caption{The maximum stiffness of 12 bovine tail vertebrae between 0 and 5000 N taken from load - displacement data. Showing the stiffness of the intact vertebrae, a post - fracture stiffness and a post - vertebroplasty stiffness for each. * Indicates those specimens that achieved a clear failure below 9500 N.}
\label{fig:allexpData}
\end{figure}

The results for the fill volume of cement in the augmented specimens is
presented in \cref{tab:cementVol} and was acquired from the down-sampled,
segmented models generated from $\mu$CT scans. It shows that fill volume varies
between 3\% and 17\% fill and in addition shows a lack of a correlation between
fill volume and increase in augmented specimen stiffness over fractured
stiffness with only five vertebrae showing an increase in stiffness. The
images in \cref{fig:t2CC2vsT8CC2} shows the extent of the cement fill for the
two vertebrae with the largest fill volume.

\begin{table}[ht]
\centering
\caption{The volume of cement and the vertebra volume for the 12 specimens used, along with the percentage cement fill and an indication as to whether the stiffnesses of the augmented vertebrae were greater than the fractured stiffness. This information was measured from the down-sampled models generated from $\mu$CT scans of the vertebrae.}
\label{tab:cementVol}

\begin{tabular}{l|>{\centering\arraybackslash}p{\dimexpr.16\textwidth}>{\centering\arraybackslash}p{\dimexpr.16\textwidth}>{\centering\arraybackslash}p{\dimexpr.16\textwidth}>{\centering\arraybackslash}p{\dimexpr.24\textwidth}}
Vertebrae & Cement Volume (mm$^3$) & Vertebra Volume (mm$^3$) & Cement Percentage of Vertebra Volume (\%) & Increase in Augmented Stiffness over Fractured Stiffness \\
& & & & \\ \hline \hline
T1 CC1 & 2260 & 32440 & 6.97 & * \\
T1 CC3 & 465 & 27039 & 1.72 &  \\
T2 CC1 & 663 & 23285 & 2.85 &  \\
T2 CC2 & 3405 & 20373 & 16.71 & * \\
T4 CC3 & 1363 & 25446 & 5.36 &  \\
T6 CC1 & 830 & 29332 & 2.83 &  \\
T8 CC1 & 1257 & 37357 & 3.36 &  \\
T8 CC2 & 4489 & 29248 & 15.35 &  \\
T8 CC3 & 1041 & 28403 & 3.67 & * \\
T9 CC1 & 2922 & 45681 & 6.40 &  \\
T9 CC2 & 2210 & 38894 & 5.68 & * \\
T9 CC3 & 2437 & 35840 & 6.80 & * \\ \hline
\end{tabular}%
\end{table}

\begin{figure}[ht]
\centering
\includegraphics[width=4in]{images/t2CC2vsT8CC2.png}
\caption{Axial $\mu$CT slices of T2-CC2 (left) and T8-CC2 (right), with cement masked in red, showing the extend of cement fill at the point where the cement was most anterior. }
\label{fig:t2CC2vsT8CC2}
\end{figure}

The attempt to reduce cement leaking through vasculature during the
vertebroplasty procedure can be seen in \cref{fig:4vert_withBlutac}. The
methods employed greatly reduced the the quantity of cement observed in both
the spinal canal and around vascular channels at the vertebral body surface
when compared to scans in \cref{fig:2vertWithoutBluTac}.

\begin{figure}[ht!]
\centering
\includegraphics[width=3.8in]{images/4vertPostBluTac.png}
\caption{$\mu$CT scans of four augmented vertebra using a steel rod to fill the spinal canal and blu-tac to cover the external vascular channels. Shows greatly reduced cement content within the spinal canal with less cement at the surface of vascular channels.}
\label{fig:4vert_withBlutac}
\end{figure}


The two plots in \cref{fig:deltaStiffness_Vs_intact} show a lack of correlation
between the difference in stiffness after augmentation when compared to both
the fractured and intact specimen stiffness and the intact stiffness. Showing
that magnitude of any increase or decrease in the vertebral stiffness following
augmentation is not caused, or a feature of the initial, intact vertebral
stiffness.

\begin{figure}[ht!]
\centering
\includegraphics[width=\textwidth]{images/deltaStiffnessVsIntact.png}
\caption{A: The difference between the post augmentation and fractured stiffness against the intact stiffness. B: The difference between the post augmentation and intact stiffness against the intact stiffness.}
\label{fig:deltaStiffness_Vs_intact}
\end{figure}



\section{Finite Element Modelling}\label{finite-element-modelling-methods}

\subsection{Introduction}

The finite element modelling of the bovine tail vertebrae, once validated,
allows investigations into various properties of the vertebrae and augmentation
process.  Such investigations include identification of geometric and material
property features, which through the use of FE models can be changed
programmatically to pinpoint their effects of certain physical scenarios.
Scenarios such as vertebroplasty, where the variation can be extended to
augmentation procedure variations, potentially leading to suggestions of best
practice depending on the properties of the vertebra in question.

Here, the main aim was to develop methods that enable the creation of specimen
specific models of bovine tail vertebrae, allowing creation and generation of
the much more clinically relevant human lumbar vertebrae using similar
methodologies.  Initially the focus was on the generation of models that
accurately describe the mechanical behaviour of intact bovine specimens, once
this was achieved to a reasonable degree an attempt to model augmented
specimens was made. In addition to these larger goals, certain sensitivity
tests were carried out, including those to understand the effects that
additional meshes, mesh sizes and mesh interactions have on model stiffness.
Finally some preliminary investigations were made into the effect of changing
augmented region positions, however the majority of this investigation will be
reserved for \cref{PCA_CHAP}.

\subsection{Model Creation}\label{model-creation}

The computational analysis of linear-elastic finite element models was carried
out using a combination the segmentation and meshing software, ScanIP
(Simpleware, Exeter, UK) and the simulation software, Abaqus (Dassault
Systemes, France). The $\mu$CT scans were converted into a finite element mesh
using the former software package, this was then imported into the second piece
of software to be configured and solved.

The scans acquired from the \(\mu\)CT scanner were converted from the ISQ file
format, generated by the scanner software, into the more portable TIFF image
format files using an existing in-house  matlab script that additionally
converts the greyscale of the scan into 256 bins. This conversion from 16 bit
TIFF files with 65,536 bins to 8 bit TIFF files was required due to the
limitation to 255 material properties within Abaqus, this allows one greyscale
value per material property (assuming all 255 greyscale values are represented
in the scan). Once the scan has been pre-processed it was imported into ScanIP
ensuring that the spacing of voxels was correctly set - in this case 82
\(\mu\)m.  Once imported, the location of the loading point was identified to
simulate the correct experimental load within ABAQUS; the marker (see
\cref{fig:expsetup}) appears bright on the scan and its centre was taken as the
load point, calculated by converting the position into mm. This was achieved by
multiplying by the native resolution of 82 \(\mu\)m.

The following parts of model creation were carried out using a Python script
from within the ScanIP software. The script carries out the process described
below and was generated by the author by carrying out the process manually and
in order to understand the steps required and then writing a script to perform
those actions. The development of the script removed much of the user variation
in the segmentation of each vertebral model. The effect of user variation
during the segmentation process is examined in \cref{sec:uvs}.

\begin{figure}[ht!]
\centering
  \includegraphics[width=4in]{images/compofDownsample.png}
  \caption{Side and top view of a vertebral $\mu$CT scan showing the effect of the downsample from 82 $\mu$m to 1mm cubed.}
\label{fig:compofDownsample}
\end{figure}




It was easier to down-sample the image stack prior to segmentation, due to the
time required for the software to generate high resolution masks and increased
memory usage at higher resolution. The effect of down-sampling can be seen in
\cref{fig:compofDownsample}. However, in certain cases, for example when
modelling vertebral augmentation, in order to attempt to capture the
intricacies of the structure and the boundaries between cement and trabecular
bone it was favourable to generate the mask prior to down sampling,
\cref{fig:fullressVPseg}. The image stack was down-sampled to voxels 1 mm
cubed, due to previous studies producing sensitivity to mesh size results that
showed a good trade off between computational cost and model accuracy
\cite{Jones2007}.



\begin{figure}[ht!]
\centering
  \includegraphics[width=4in]{images/fillTheVoid.png}
  \caption{Side view of a vertebral model showing segemented vertebra, including the internal void that is filled.}
\label{fig:fillTheVoid}
\end{figure}


Once down-sampled the image stack was segmented into the constituent parts -
the vertebrae and cement end-caps. The different regions that were required to
be segmented have different greyscale values, hence the general shape of the
masks was created through a thresholding tool that selected volumes of the
image stack between two greyscale bounds. For the end-caps these bounds were usually
between greyscale values of 12-22 and, if the specimen was not augmented, the
vertebrae between 23-255. For augmented specimens these limits change to 23-65
for the vertebrae and 66-255 for internal cement containing barium sulphate.
These values were selected by visually limiting the amount of unwanted material
selected within the threshold and maximising the wanted material, for example -
selecting as much of the end-caps as possible while limiting the selected
background and vertebral material to a minimum. This thresholding can be seen
in \cref{fig:fillTheVoid}.  It was preferable to avoid internal voids within
each mask, due to the potential for errors that may arise from the extra
surfaces and contacts created when solved in ABAQUS, these were removed with the use of the morphological
close and cavity fill tools within ScanIP (\cref{fig:fillTheVoid}).





The following parts of the method were carried out manually, following the
completion of the automatic segmentation python script. The two end-caps were
separated into two separate masks by first duplicating the mask and then flood
filling each end-cap to form separate masks.

An FE model was created using the previously generated masks and properties for
the volume meshing, materials and contacts were set. The grid size for the
model was set to 1 $\times$ 1 $\times$ 1 using the FE grid algorithm which uses
a mix of tetrahedral and hexahedral elements. Material properties were set to
homogenous with a Young's modulus = 2.45 (GPa) and Poisson's ratio
= 0.3 for the end-caps. Material properties for the internal cement regions for
the augmented vertebrae are described later. The material properties for the vertebral volume were set to a
greyscale based material type using the greyscale background information. The
coefficients were set so that both the density and Young's modulus were equal
to the greyscale value for that element, allowing the Young's modulus to be set
correctly in following steps and as described in Section \cref{material-properties-bov}.

Contacts were set as placeholders to be edited in Abaqus in the steps
following. These were contact pairs between each component and another between
the superior end-cap and the upper boundary on the Z-axis. The second contact
type was a node set between the inferior end-cap and the lower boundary on the
Z-axis used to create an enastre boundary condition for the model base.

\begin{figure}[ht!]
\centering

  \includegraphics[width=4in]{images/abaqus_side_view_Both.png}
  \caption{Side \& top down view of a vertebral model showing the alignment of the analytical rigid plane.}
\label{fig:abaqus_top_view}
\end{figure}


Following this, the FE model was meshed, then exported into an INP file format.
This file was imported into Abaqus where the following configuration was
completed by a second python script. An analytical rigid plate was created to
represent the upper loading platen of the materials testing machine and was
centred at the loading point previously found from the marker, this can be seen
in \cref{fig:abaqus_top_view}. Once aligned any previous placeholder
interactions were removed and a tied interaction was created between the rigid
plate and the superior end-cap, along with tied interactions between the
vertebra and both end-caps and, if appropriate, to any internal cement volumes.
An encastre boundary condition was created at the bottom surface of the
inferior end-cap removing all rotational and translational movement and
therefore mimicking the experimental setup. A displacement boundary condition
was applied to a reference node at the centre of the rigid plate and therefore
loading position, the properties were set such that 1 mm of displacement occurs
in the negative Z direction; lateral motion in the X and Y planes was
restricted, while rotation about the loading point was allowed - 
mimicking the experimental steel ball setup.

The python script was written to set the material properties of the greyscale
dependant vertebral elements by setting the Young's modulus to the greyscale
value multiplied by a conversion factor (which is discussed in section
\cref{material-properties-bov}). The script allowed Abaqus to solve the models
and outputs the stiffness for each model. This was calculated by the dividing
the axial reaction force (axis of load application) at the reference point by
the  1 mm of displacement.

\subsubsection{Augmented Model Generation}

In order to capture the detail of interdigitation between the
vertebrae and the injected cement, the masking process was carried out prior to
downsampling, seen in \cref{fig:fullressVPseg}. If masked post-down-sample it
became difficult to define the cement boundaries and the masked volume was
inaccurate when compared to the full resolution scan. Masking the internal
cement region used the same thresholding approach described above.


\begin{figure}[hbt]
\centering

  \includegraphics[width=10cm]{images/fullres_VP_segment.png}
  \caption{A lateral slice through an augmented bovine tail vertebra, showing
the cement mask in red at the full 82 $\mu$m resolution.}
\label{fig:fullressVPseg}
\end{figure}


\subsection{Material Properties}\label{material-properties-bov}

\subsubsection{Bone Material Properties}

Material properties for the bone tissue were modelled elastically using a bone
element-specific elastic modulus (\(E_{\text{ele}})\) that is dependent on the
average greyscale value for the element in question
(\(\text{GS}_{\text{ele}})\) with the conversion factor between the two being
$\alpha$.

\[E_{\text{ele}} = \alpha\ \text{GS}_{\text{ele}} (GPa)\]

This was required due to each element containing differing quantities of bone
and bone marrow due to the continuum level modelling carried out. Hence, a
homogenous value for the trabecular bone would not have represented the varying
average of material properties seen in each element. The conversion
factor, $\alpha$, was used to convert between the greyscale value for each
element and the Young's modulus.

A separate set of 24 bovine tail vertebrae underwent the same experimental and
computational methods described previously in order to tune the value of
$\alpha$. This
additional set was split into two groups of 12 and was worked on in collaboration with Dr Sebastien Sikora, Dr
Fernando Zapata Cornelio \& Ruth Coe (Research Fellow, Research Fellow \& PhD
student respectively).  The groups consisted of a calibration group (used to
determine the value of $\alpha$ and a validation group. For the validation
group material properties were assigned - multiplying the greyscale for each
element by \(\alpha\) prior to compressing the model by 1 mm in Abaqus and was
used to validate against the experimental values of stiffness.

The calibration for \(\alpha\), the conversion factor was carried out using a
golden section search scalar optimisation process. Specifically using the Brent
method within the opti4Abq toolbox \cite{Mengoni2017}. The
objective of this toolbox was to find the root mean square normalised
difference between the experimental specimen stiffness and the finite element
stiffness and iterate until the objective function achieved a value of
10\textsuperscript{-3}.

\subsubsection{Augmented Specimen Material
Properties}\label{augmented-specimen-material-properties}


For convenience the values for the Young's modulus for the interior cement
volume were initially set to that of the inferior and superior end-caps.
However, from previous studies into modelling bone cement interactions
\cite{Sikora2013a, Kinzl2012a, Kinzl2012, Chevalier2008, Wijayathunga2008,
Zhou2000, Tozzi2012, Janssen2008}, simple material properties of PMMA do not
accurately describe the environment.  Due to the rule of mixtures and the
results found in the literature \cite{Kinzl2012a,Race2007}, the effect of
reducing the Young's modulus was investigated. This was carried out by reducing
the Young's modulus in 10 percent increments from a value of 2.45 GPa to 1.225
GPa.

Additionally, a preliminary investigation was carried out, identifying the
effect of a yielding material interface between the bone and the cement,
following the work by Sikora \cite{Sikora2013a}.  Here, a small interface layer
(1 mm in thickness due to the model resolution) represents buckling that occurs
in the trabeculae partially captured in the cement volume.  This investigation
identified the effect of different yield stress values in combination with
different Young's moduli for both the interface and the main cement volume.

The yielding region was created through the duplication and then erosion of the
cement mask within Simpleware ScanIP, shown in \cref{fig:interfacecreation}.
This gave two masks, where one acts as the internal cement volume and the other
as an interface layer.  Generic material properties were set to the region
within ScanIP, allowing the values of the yield stress and Young's moduli
values to be changed within the python setup script.  The properties tested
included changing the yield stress and Young's modulus of the region, whilst
maintaining the properties of the internal cement volume at 1.225~GPa.  Values
for the yield stress in the interface region were varied between 1 MPa and
0.001~MPa, with the material become perfectly plastic beyond this.  Young's
modulus values for the interface region were varied between 1.225~GPa and
0.1~GPa. These values were manually tuned to find the optimum values with
respect to the CCC of the set.

\begin{figure}[ht!]
\centering
\includegraphics[width=10cm]{images/interface_creation.png}
\caption[Depiction of the yielding material interface creation]{Creation of the interface layer: A, showing the initial description of the cement region in pink and B, showing the interface in pink and cement region in red, following the duplication and erosion.}
\label{fig:interfacecreation}
\end{figure}

\subsection{Sensitivity Tests}\label{sensitivity-tests}

\subsubsection{Mesh Size Sensitivity}\label{mesh-size-sensitivity}

Element sizes of 1 x 1 x 1 mm were used throughout, following previous
convergence studies on porcine vertebrae \cite{Jones2007}. The results of the
convergence study on porcine vertebrae showed that reducing the element size
below 2 x 2 x 2 mm led to changes in the model that were smaller than predicted
errors originating from other factors, such as experimental errors and the
simplification of boundary conditions.  However, reducing the element size to 1
x 1 x 1 mm allows greater resolution when modelling the intricacies of the
cement mesh for augmented specimens, the difference between the two resolutions
can be seen in \cref{fig:vertslice}.

\begin{figure}[ht!]

\centering
\includegraphics[width=3.90994in,height=1.80208in]{images/res_comp.png}
\caption{Mid-slice through an augmented vertebra, cyan: vertebral body, red:
cement. A, element size of 1 x 1 x 1 mm. B, element size of 2 x 2 x
2 mm.}
\label{fig:vertslice}
\end{figure}


\subsubsection{Sensitivity to an Additional
Mask}\label{sensitivity-to-an-additional-mask}

The addition of cement into the vertebral body created an extra mesh boundary
within the mesh containing the vertebral elements. In order to test what effect
this may have on the stiffness of models containing an extra mesh boundary, an
un-augmented specimen was tested with an extra mesh representing the cement,
but with the material properties of its elements set based on their greyscale
as with the other bone elements.  The mask was created by duplicating the
vertebral mask and eroding it until the volume was approximately 20 \% of its
original volume. This allowed testing to be carried out on the effect of the
extra mesh alone, while using an augmented specimen would allow a more accurate
cement shape, it would hinder setting material properties to that of the bone
greyscale and create an additional level of uncertainty. Mesh interactions
between the two meshes (internal vertebral surface and the cement mask surface)
were set using the contact pair interaction and treated similarly to the
interaction between the end-caps and the vertebrae. Following model setup in
ABAQUS as outlined in section \ref{model-creation} the model was loaded in
compression to 1 mm and its stiffness was recorded.

There was no difference between the two models, with and without the internal
cement mesh, meaning that any changes to the augmented model stiffness was due
to the material properties of the cement.

\subsubsection{Mesh Interactions}\label{mesh-interactions}

The effect of mesh interaction between the vertebral body and the internal
cement mesh were tested by comparing a) tied interactions between the two
surfaces and b) removing any interaction and merely changing the material
properties of the internal cement region (neglecting the contact pair steps
described earlier). This was carried out for four augmented specimens following
the same setup within ABAQUS as described earlier.

The results can be seen in \cref{tab:meshint}, showing a negligible difference
between tied and non-tied (single mesh) models. This difference falls well
below the difference between experimental and computation, especially for the
augmented specimens, hence the effect of this interaction can be neglected from
further test.

\begin{table}[ht!]

\caption{The difference between interaction properties, tied and not tied for
four augmented vertebrae specimens.}
\label{tab:meshint}
\centering
%\resizebox{\textwidth}{!}{
  \begin{tabular}{c|c|c|c}
Vertebrae (Tail Number,  & Tied Interaction  & No Tied Interaction  &
Difference \\
Vertebral Level) & (N/mm) & (N/mm) & (\%) \\
\hline
\hline

T2 CC1 &	 5496 &	 5496 &	 0\\
T2 CC2 &	 8086 &	 8086 &	 0.001\\
T6 CC1 &	 3686 &	 3686 &	 0.001\\
T4 CC3 &	 6059 &	 6059 &	 0.0005\\ \hline

\end{tabular}
%}
\end{table}



\subsubsection{Augmentation Location Sensitivity}

Another preliminary study was carried out (preliminary given the limited nature
of the agreement between computational and experimental results and aims to
improve this with the use of human tissue) to identify the sensitivity of the
models to the position of the cement volume.  Tests were carried out to
identify the effect of moving a 12 \% fill cement volume axially and sagittally
in 2 mm increments.  This utilised the yielding interface described in
\cref{augmented-specimen-material-properties} along with the +CAD tools built
into Simpleware ScanIP to move a surface based description of the cement volume
in the two anatomical planes.  A sphere surface of volume equivalent to 12 \%
fill volume for the T12 CC2 vertebra was created within the +CAD software and
imported into the project file for the vertebra, where it was positioned in
it's first position.  This first position was 1 mm from the top of the
vertebral body, ensuring 1 mm of bone surrounded the sphere.  It was then
converted into a mask, duplicated and eroded in the same method described in
\cref{augmented-specimen-material-properties} to create the yielding material
interface and set with optimum material properties.  Following this the mesh
was generated and the abaqus input file was exported.  The sphere surface was
then moved 2 mm in the inferior direction, the previous masks deactivated and
the process repeated through the use of a python script to carry out the
operations until the bottom of the vertebra had been reached.  This was also
carried out in the sagittal plane. 


\begin{figure}[ht!]

\centering
\includegraphics[width=4in]{images/xaxis_mv_cmt.pdf}
\caption{The effect of moving the cement volumes from the most anterior position to the most posterior position on the recorded stiffness of the vertebra, when using uniform 12 \% fill volume of cement. }
	\label{fig:xaxis_mv_cmt}
\end{figure}

\begin{figure}[ht!]

\centering
\includegraphics[width=4in]{images/yaxis_mv_cmt.pdf}
\caption{The effect of moving the cement volumes from the most superior position to the most inferior position on the recorded stiffness of the vertebra, when using uniform 12 \% fill volume of cement. }
	\label{fig:yaxis_mv_cmt}
\end{figure}

\begin{figure}[ht!]

\centering
\includegraphics[width=3in]{images/mv_cmt_image.png}
	\caption{A density map of the T12 CC2 vertebra. Yellow/orange elements are the most dense, while blue elements are the least. }
	\label{fig:mv_cmt_image}
\end{figure}

The results of moving the cement volumes through the T12 CC2 vertebra are shown
in \cref{fig:yaxis_mv_cmt,fig:xaxis_mv_cmt}.  For the sagital plane a reduced
stiffness is seen when the volume of cement is positioned most anterially,
where the yielding interface of the cement volume encroaches on the denser bone
of the cortical shell.  Centrally the stiffness is greatest, where the cement
volume replaces the least dense bone, which can be seen in
\cref{fig:mv_cmt_image}, where the least dense (blue) parts are ``hidden" by
the cement volume in certain positions.  Little change to the stiffness at the
posterior side of the vertebra is seen due to the additional support from the
posterior elements.

Axial movements of the cement volumes show that the most inferior and most
superior positions resulted in the greatest stiffness and central positions
resulted in the least stiff models.  This is potentially due to the greater
quantitiy of bone surrounding the cement volumes at top and bottom where the
vertebral body is wider.  The peak in the central loading positions is again likely
due to the ``hiding" of the weaker bone in the centre of the vertebra.

Finally, despite the changes to the stiffness when moving the cement volume,
these changes are smaller than five percent between the most and least stiff
result, suggesting cement volume movement at these small volumes is not
significant.  It may also suggest that cement shape may play a more important
role, where endcap to endcap distributions show much larger effects as shown in
\cite{aquarius2014prophylactic, steens2007influence, Chevalier2008}.

\subsubsection{User Variability Sensitivity}
\label{sec:uvs}

A user variability study was conducted to identify the variation in the
modelling approaches, specifically masking the bone and endcap regions. Four
users each masked the bone and endcaps for eight vertebrae using thresholding
and other morphological filters (which would later be automated using the best
approach as described previously). The FE models were then generated and
the material properties were optimised using the greyscale optimisation process
separately for each users models.

The results for the variability can be seen in \cref{tab:user_var} showing the
mean, maximum, minimum and range as well as the mean for each. The mean range
of values was 158, with a mean model stiffness across the four users of 1928
N/mm. The maximum difference between users was for specimen six where a range
of 246 was found. This equates to a maximum possible error of 14 \% for
different users carrying out the same models creation process.

\begin{table}[h]
    \centering
    \caption{The variability the modelling approaches of four users with 8
        models, each users models undergoing separate greyscale optimisation
    and FE model solving}
\begin{tabular}{c|c|c|c|c}
         & \multicolumn{4}{c}{Stiffness (N/mm)} \\ 
Specimen & Mean  & Maximum  & Minimum  & Range  \\ \hline
1        & 2457  & 2567     & 2379     & 188    \\
2        & 2298  & 2368     & 2218     & 150    \\
3        & 2374  & 2463     & 2261     & 202    \\
4        & 1600  & 1659     & 1545     & 114    \\
5        & 2281  & 2324     & 2218     & 106    \\
6        & 1899  & 1990     & 1744     & 246    \\
7        & 1330  & 1398     & 1236     & 162    \\
8        & 1183  & 1219     & 1119     & 100    \\ \hline
Mean     & 1928  & 1998   & 1840     & 158 \\ \hline
\end{tabular}
    \label{tab:user_var}
\end{table}

While 14 \% error between users if relatively high, the models used in this
chapter and subsequent chapters were all build by the author. Users tended to
produce consistent differences in the stiffness of generated models, for
example user one had consistently lower stiffness values and user three had
consistently higher stiffness values (means of 1876 N/mm and 1952 N/mm
respectively). This, in addition to the fact that the models are built using a
python script within Simpleware ScanIP, means that the single user variability
should become zero.

\subsection{Result Analysis}

The statistical approach to quantifying the agreement between the computational
results and the experimental results for the measured stiffness uses the
concordance correlation coefficient (CCC).  CCC measures the agreement between
two variables and is described by Lin \cite{lawrence1989concordance} as a
method to evaluate reproducibility.  The CCC is calculated as:

\begin{equation*} CCC = \dfrac{2 \rho \sigma_x \sigma_y}{\sigma_x^2 +
\sigma_y^2 + (\mu_x - \mu_y)^2} 
\end{equation*} 

Where $\mu_x$ and $\mu_y$ are the means for the two variables and $\sigma_x^2$
and $\sigma_y^2$ are the corresponding population variances.  $\rho$ is the
Pearson correlation coefficient between the two variables.  For $n$ independent
pairs of samples:

\begin{equation*} CCC = \dfrac{2S_{12}}{S_1^2 + S_2^2 +
(\bar{Y_1}-\bar{Y_2})^2} 
\end{equation*}

Where, 

\begin{equation*} \bar{Y_j} =
\dfrac{1}{n} \sum_{i=1}^{n} Y_{ij},\ S_j^2 = \dfrac{1}{n} \sum_{i=1}^{n}
(Y_{ij} - \bar{Y_j})^2,\ j = 1, 2; 
\end{equation*}

and

\begin{equation*} S_{12}
    = \dfrac{1}{n} \sum_{i=1}^{n} (Y_{i1} - \bar{Y_1})(Y_{i2} - \bar{Y_2})
\end{equation*}

Specifically, this quantifies the degree to which pairs sit on the $x=y$ line.
Departures from this perfect agreement line result in a $CCC < 1$ even in cases
where the Pearson correlation coefficient would equal 1.  $CCC = 0$ corresponds
to no agreement and a value of $-1$ would be perfect negative agreement.



\subsection{Results} \label{bov:results}

The optimisation process gave a value for the conversion factor of 0.012529,
allowing conversion between greyscale values for elements and their elastic
modulus. This value was used for the bone constituents of the intact and
augmented vertebrae presented in \cref{fig:compvexpscatter}, which shows the
agreement between the \textit{in vitro} and \textit{in silico} results for the
specimen specific models. The agreement of intact vertebrae achieved a
concordance correlation coefficient (CCC) of 0.49 compared with 0.14 for the
augmented vertebrae (simple tied interaction) (\cref{tab:int}).  The
non-augmented CCC value increased to 0.60 if the uncharacteristically stiff
T8-CC2 was removed.

\begin{figure}[ht]
\centering
\includegraphics[width=\textwidth]{images/exp_vs_comp_both.png}
\caption{The \textit{in silico} compared with \textit{in vitro} stiffness for intact specimens (triangles) and augmented specimens (circles). The dotted line shows a one-to-one correspondence.}
\label{fig:compvexpscatter}
\end{figure}

\begin{table}[ht]
\centering
	\caption{The mean, standard deviation and concordance correlation coefficient (CCC) of the intact and augmented vertebrae (simple tied interaction) for \textit{in vitro} and \textit{in silico results}.}
\label{tab:int}
\begin{tabular}{l|c|c|c}
     Intact Specimens     & Mean Stiffness & Standard Deviation & CCC                     \\ \hline \hline
\textit{in vitro}  & 5684 & 1196             & \multirow{2}{*}{0.4895} \\
\textit{in silico} & 5610 & 958               &                        \\
\hline
 Augmented Specimens
 \\ \hline \hline
\textit{in vitro}  & 4246 & 1371               & \multirow{2}{*}{0.14} \\
\textit{in silico} & 6507 & 1298               &                      \\ \hline
\end{tabular}
\end{table}

The effect of changing the modulus of the cement volume in the augmented
specimens is presented in \cref{fig:redEModBar}. There was a linear decreases
in the stiffness of vertebrae with the reduction of the elastic modulus for the
internal cement volume. The two vertebrae that show more prominent decreases in
stiffness were those vertebrae that contained larger volumes of cement
following their augmentation.

The effect that this has on the data with regard to the \textit{in vitro}
stiffness results can be seen in \cref{fig:redEModscatter}, where the reduction
in \textit{in silico} stiffness moves the data points closer to the $x = y$
line of perfect agreement between the experimental and computational results.
This gave a CCC of 0.18.

The increasing agreement with increasing sophistication of
modelling the augmentation can be seen n \cref{fig:exp_50_yield_tied},
comparing the simple tied interaction, the reduced modulus approach and the
combination of this and the yielding material interface.  The CCC improves when
using the reduced modulus at 50 \% (1.225 GPa) from the simple tied interaction
with an improvement to 0.18 from 0.14.  Another improvement is seen when
combining the reduced modulus for the cement with the yielding interface
(Young's modulus 1.225 GPa, yield stress 0.005 MPa), with the CCC improving to
0.23.  The effect of changing these interactions and properties is not uniform,
also shown in \cref{fig:redEModBar}, the vertebral models respond differently
to different interactions, with increases and decreases seen within the same
set of vertebrae when changing the
same interaction.

\begin{figure}[ht!]
\centering
\includegraphics[width=\textwidth]{images/reductionOfEMod_Bar.png}
\caption{The percentage decrease in the vertebral stiffness after reducing the elastic modulus of the cement volume within 12 augmented vertebrae.}
\label{fig:redEModBar}
\end{figure}

\begin{figure}[ht!]
\centering
\includegraphics[width=5.3in]{images/reductionOfEMod_Scatter.png}
\caption{The effect of reducing the elastic modulus of the cement volume within 12 augmented vertebrae. Shows the in silico stiffness for the six elastic moduli tested against their in vitro stiffness. }
\label{fig:redEModscatter}
\end{figure}

\begin{figure}[ht!]
\centering
\includegraphics[width=5.3in]{images/exp_50_yield_tied.pdf}
	\caption[A comparrison of the different approaches to modelling augmentation in the bovine tail vertebrae.]{A comparrison of the different approaches to modelling augmentation in the bovine tail vertebrae. Comparing a simple tie between the bone and cement, a reduced modulus for the cement region and a reduced modulus in combination with a yielding interface layer between the bone and cement.}
	\label{fig:exp_50_yield_tied}
\end{figure}

\section{Discussion}

\subsection{Experimental Discussion}

The experimental section aimed to develop the various aspects of the
experimental methodology on a set of bovine tail vertebrae, allowing for an
easy transition into using human lumbar vertebrae, which is discussed in the
following chapter.  Previously developed methods of $\mu$CT scanning have been
verified for the bovine tail vertebrae.  Methods of performing vertebroplasty
and dealing with the unique challenges arising from the bovine vertebrae have
been created.  The experimental part of this chapter provides a good basis for
both the continued modelling of vertebroplasty (especially modelling the cement
- trabecular interface) and to continue
the experimental work using human tissue.  Understanding the challenges of
vertebroplasty, will be invaluable when transitioning
onto the much more limited source of human vertebrae.

%^ reiterate aims of exp work

Regarding the results of stiffness at the intact, fractured and augmented
stages, the expected trends were not always clear.  Most commonly the intact
vertebrae have the greatest stiffness with the fractured stiffness showing a
reduced value following the damage created with the initial load to failure.
The variation of the decrease (and increase) in stiffness for the fractured
vertebrae may have a variety of reasons, although the most likely cause is
level of damage caused in the initial ``load to failure''.  These tests varied
between the typical load displacement that includes a failure
(\cref{fig:failure_non_failure}:A) and those that show no sign of failure up to
the limit of the load cell (\cref{fig:failure_non_failure}:B).  It is difficult
to observe any correlation between these vertebra that showed clear failure
(T2-CC1 and T2-CC2), those that reached 9500 N and a reduction in the fractured
stiffness.  It is not to say that the vertebra that reached 9500 N experienced
no damage, with the gradient of the load displacement curve often reducing and
plateauing as the 9500 N limit approached.  The interesting increase in the
fractured stiffness for T1-CC1 compared to the intact stiffness may be
explained if it is assumed that the compacted trabeculae following the first
load to 9500 N result in a stiffer material for the following tests. Other
possibilities that may explain the increase in stiffness include a change in
the seating of the vertebra within the PMMA endcap. If the initial load changed
how the vertebra was positioned in the endcaps it would change the response to
loading, even when loaded at the same point.

The cement fill volume information shows that a small percentage of cement is
injected into the vertebrae on average, with only two vertebrae approaching the
clinically relevant 20\% fill.  Unexpectedly, only one of these two vertebrae
showed an increase in augmented stiffness over the fractured stiffness.  A
possible explanation is that it is not only the fill volume that is important
in restoring the vertebral stiffness but the placement or shape of the cement
fill volumes.  This is shown when comparing the segmented scans of the the two
vertebrae with the greatest fill volume with the T2-CC2 specimen showing cement
extending to the anterior wall of the vertebral body, while the cement is
limited to the posterior and centre of the vertebral body for T8-CC2.  This may
help to explain why the stiffness of T8-CC2 did not increase following
augmentation.  The reduction in stiffness following augmentation for seven of
the twelve vertebrae may be due to damage caused by the insertion of
vertebroplasty needles.  Clinically this damage left behind from the needle
channels would heal, most likely restoring the stiffness of the vertebrae back
to its intact properties. Additionally, needle insertion into the dense bone of
the bovine tail vertebrae required a mallet to reach the anterior portion of
the vertebral body, clinically this would not be required due to the much less
bone of the human lumbar vertebrae
\cite{aerssens_jeroen_boonen_steven_geert_dequeker_1998}. Such a
violent approach would damage the bound surrounding the needle entry, causing
micro-fractures not visible on the $\mu$CT scans.

Another possible area of inconsistency is the temperature at which the vertebra
were mechanically tested, while it is ensured that the specimens were fully
defrosted they were tested at both fridge temperature (4$^\circ$C) and room
temperature (20$^\circ$C).  The effect of this variation in temperature needs
to be identified or a consistent temperature needs to be used for subsequent
tests.

Despite encouraging results regarding the vertebroplasty methodologies it was
difficult to achieve the desired quantity of the cement in the vertebral body.
This was mainly due to the difficulty injecting the cement in a smooth manner,
which may have been caused by either the tip of the needle becoming blocked
following its reinsertion into the needle track, more viscous marrow stopping
the displacement of less viscous marrow by the cement or compacted trabeculae
around the needle channel that limit the flow of cement past them.  One option
to test in future work would be side opening needles, which would help guide
the cement more accurately to the regions required while circumventing issue
with the needle becoming blocked.  Additionally, the shape of the vertebrae
adds to the difficulty of performing vertebroplasty, with the narrow
cylindrical shape of the vertebrae meaning the cement has to travel large
distances axially through the trabecular bone to achieve clinically relevant
cement fill volumes.  In contrast the much wider human lumbar vertebrae used in
latter chapters give much more space to inject larger volumes of cement.  This,
multiplied with a greatly reduced average trabecular density for the
osteoporotic human vertebrae (see \cref{fig:bmdgraph} and
\cite{aerssens_jeroen_boonen_steven_geert_dequeker_1998,wilke1997sheep} should
allow for greater volumes of cement to be injected. 

When attempting to identify trends between the pre and post augmented and pre
and post fractured vertebral stiffness, no relationships were found.  It was expected that
initially weaker vertebra would show a larger change in stiffness between the
fractured state and the augmented state and similarly between the intact and
augmented states.  However, no such trends were found, suggesting that
properties other than the stiffness or bulk material properties are the cause
of the variation.  Properties such as the vertebral geometry and trabecular
structure, which are investigated through the use of statistical shape models
in \cref{PCA_CHAP}.

The experimental methods currently developed will be of great value when
starting experimental work using human tissue albeit many will require adaption
due to the differences between the tissue types.  These include the density of
the bone and methods of inserting the needles, where with the available human
tissue being from the elderly, the bones will be most likely be osteoporotic.



\subsection{Computational Results Discussion}

The aims of the computational part of this chapter were to develop methods to
build models of vertebrae, using automated approaches and carrying out
sensitivity tests to identify the best methods to use when carrying out similar
model creation on the set of human lumbar vertebrae in \cref{Chapter_HT}. This
was achieved, with automation of the segmentation process within ScanIP and
automation of the model setup within Abaqus achieved using a series of python
scripts. These scripts require no conversion to be used with the modelling of
human lumbar vertebrae, with the exception of the thresholding carried out in
scanIP. The preliminary sensitivity tests give a first idea of what variations
the vertebrae are sensitive to and will allow further investigation in
subsequent chapters.


The computational methods developed and results acquired, as with the
experimental results, provide a good base for using human vertebrae and for the
continued development of augmented vertebra modelling. The methods and
automation of the creation of models both greatly reduce the time spent on
model generation and reduce human error. This allows a relatively easy
translation to human vertebrae, with only minor adjustments to the thresholds
that define materials.

The current methods of masking and meshing the internal volumes of cement in
the augmented models provides a good method of creating augmented models. The
sensitivity tests carried out on the additional mask inside the vertebrae shows
that any effects seen are due to the volume of cement and not a simulation
problem. Similarly the inclusion of a tied interaction between the two meshes
failed to affect the result, something especially useful when considering
alternative mesh interaction to model the cement - trabeculae interaction.

Augmented volume location sensitivity tests suggests little influence of the
position of the cement volume on the reported stiffness when loaded.  It does
highlight a potential source of error, where anteriorly placed volumes of
cement would be expected to increase the stiffness of the vertebrae, given the
reduced support on the anterior side.  However, due to the yielding interface
replacing the dense cortical shell, the result is a reduced stiffness.
Therefore care needs to be taken when investigating cement volume movement as
to not create unnatural simulations.  When modelling experimentally augmented
models this is not a problem as the cement does not replace the bone
experimentally and therefore the cortical shell is preserved when modelling the
vertebra.  Additionally, moving the cement volumes highlights the importance of
filling or replacing the weakest, least dense parts of the vertebrae with
cement.  Although in this can the change of replacing the central ``void" in
the vertebra is small compared to placing the cement volume elsewhere, as
cement volume size increases (and modelling accuracy improves) the effect may
become more pronounced.  This is investigated further in \cref{PCA_CHAP}, where
cement volumes are moved within statistical shape models of osteoporotic human
lumbar vertebrae, which exhibit similarities to the voids found in the bovine
tail vertebrae.

The intact models agree well with experimental results, with very similar
results for the mean and, excluding anomalous results, a good CCC value,
showing that the previously validated conversion factor works well with this
set of data and that segmentation and model setup works correctly. The poor
agreement with the augmented specimen models and their experimental
counterparts was an expected result that agrees with similar studies in the
literature - \cite{Wijayathunga2008}. In an attempt to produce a better
agreement, the elastic modulus of the cement volume was reduced in accordance
with the experimental results of Race et al. \cite{Race2007} and similar
methods employed by Wijayathunga et al. \cite{Wijayathunga2008}, where the
reduction in modulus is expected due to the greater ratio of monomer to powder
used, gaps between the bone and cement and pores within the cement. The
reduction in stiffness forms a linear pattern as the elastic modulus is
reduced, with those vertebrae that show the greatest reduction being those
containing the largest volume of cement. While these results do show a
reduction in the stiffness, closer to that of the experimental values, it does
not explain the disagreement fully. This suggests that a combination of
improvements to the augmented models is required.

Future work will utilise the results acquired, especially those relating to the
cement modulus and the cement - bone interactions, to understand how to model
this interface more realistically and achieve good agreement between
experimental and computation results of stiffness for augmented specimens.


