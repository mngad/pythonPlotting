\chapter{Alternate Methodologies} 

\section{Inverse Greyscale Material Properties for Augmented Region} \label{sec:invGS}

A different approach to modelling the cement region was also attempted.  This
used an inverse greyscale material property for the cement region, rather than
a homogeneous material property for the region, based on the greyscale
background of the non-augmented vertebral scans with the mask from the
augmented models overlaid.  This meant that darker regions of greyscale (within
the cement volume) would be given a Young's modulus of PMMA cement (2.4 GPa)
which would decrease linearly to the Young's modulus of the bone when entering
into the brighter regions of the greyscale background.  Representing the cement
regions using this method meant that the underlying bone present inside the
cement volumes is not ignored, instead the interdigitation between bone and
cement in represented.  While this representation is potentially inaccurate, it
provides more detail that using a homogeneous material property and more detail
than van be seen on the augmented vertebral scans.  The relationship for the
cement volume using this method was therefore:

\begin{equation}
	E(\rho) = \alpha \rho + c
\end{equation}
\begin{equation}
	E(\rho) = -0.00842 \rho + 2.4
\end{equation}

Where $\rho$ is the greyscale value of each voxel and $E$ is the Young's
modulus of each element within the cement region.  This ensured that the
Young's modulus of the regions containing no bone (greyscale = 0)  was equal to
that of the cement and regions containing bone (greyscale = 255) had a Young's
modulus equal to 250 MPa, an approximate value for trabecular bone, with a
linear relationship between.


