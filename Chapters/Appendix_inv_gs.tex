\chapter{Alternate Methodologies} 

\section{Inverse Greyscale Material Properties for Augmented Region} \label{sec:invGS}

A different approach to modelling the cement region was also attempted.  This
used an inverse greyscale material property for the cement region, rather than
a homogeneous material property for the region, based on the greyscale
background of the non-augmented vertebral scans with the mask from the
augmented models overlaid.  This meant that darker regions of greyscale (within
the cement volume) would be given a Young's modulus of PMMA cement (2.4 GPa)
which would decrease linearly to the Young's modulus of the bone when entering
into the brighter regions of the greyscale background.  Representing the cement
regions using this method meant that the underlying bone present inside the
cement volumes is not ignored, instead the interdigitation between bone and
cement in represented.  While this representation is potentially inaccurate, it
provides more detail that using a homogeneous material property and more detail
than van be seen on the augmented vertebral scans.  The relationship for the
cement volume using this method was therefore:

\begin{equation}
	E(\rho) = \alpha \rho + c_{cement}
\end{equation}
\begin{equation}
	E(\rho) = -0.00842 \rho + 2.4
\end{equation}

Where $\rho$ is the greyscale value of each voxel and $E$ is the Young's
modulus of each element within the cement region.  This ensured that the
Young's modulus of the regions containing no bone (greyscale = 0)  was equal to
that of the cement (approximately 2.4 GPa) and regions containing bone (greyscale = 255) had a Young's
modulus equal to 250 MPa, an approximate value for trabecular bone, with a
linear relationship between. Brief optimisation of the constant using values of
between 0.5 and 5 with 0.5 increments was also carried out.

The results of using this approach were poor, when using the material
properties as described above on their own and when incorperating these
material properties into the yielding interface modelling methods. With the
new material properties along, a negative CCC was achieved with R$^2$ values of
$\leq$ 0.1. Using these material properties in combination with the yielding
interface the best CCC achieved was 0.33.

The reasons for the poor agreement are difficult to determine and potentially
require further testing and optimisation. The introduction of an intermediate
later between elements classed as bone and elements classed as cement, would
potentially describe the shrinkage that occurs when the cement sets and other
voids that form inside the augmented region. This would create properties
similar to the approach used by Kinzl et al. \cite{kinzl2013experimentally},
where random spheres were inserted into the augmented region to represent the
voids that form. Equations for this approach would be in the form:
\begin{equation}
	E(\rho)= 
	\begin{cases}
	\alpha \rho + c_{cement}, & \text{for}\ \rho \leq 50\ \text{and}\ \rho \geq 100 \\
	\beta, & \text{otherwise}
	\end{cases}
\end{equation}

Where $\beta$ is the Young's modulus on the intermediate region between the
bone and the cement, forming through the cement shrinking.
