\chapter{Human Tissue}

\section{Introduction}


\section{Methods}

\subsection{Potting}

The geometry of human lumbar vertebrae varies considerably to that of the bovine tail vertebrae from which this methodology is based.
This is characterised by much larger posterior elements with the facets extending much lower, below the bottom of the vertebral body.
Hence, to correctly pot the human vertebrae much more cement must be used, especially for the posterior end-cap, in order to cover the bottom of the vertebral body and the extending posterior elements.
This means that much more of the posterior elements are constrained, therefore restricting the rotation of the vertebral body endplates under axial load.
In addition to this the larger posterior elements which are captured within the PMMA end-caps will transmit load and take a greater share of the load when compared to the bovine tail vertebrae.
Given that vertebroplasty attempts to restore the stiffness of the vertebral body and that there is no understanding of specifically how the loads are shared between the vertebral body and posterior element, this presents a problem.

A solution to this is to remove the posterior elements, following such methods as \cite{Wijayathunga2008,RobsonBrown2014}, where only the vertebral body is modelled.  
This allows the stiffness of the vertebral body alone to be captured and modelled.
The posterior elements were removed by cutting through the pedicles at the narrowest part, limiting damage to the region.

To pot the specimens that now lack a spinal canal, a retort stand was used to hold the vertebra, ensuring that both endplates were level on average.
The specimen was then lowered down into the potting container leaving 5 mm between the bottom of the vertebra and the container.
PMMA was poured into the container until the entire of the endplate was touching cement, with the edges of the vertebral body covered.
Care needed to be taken to ensure all of the endplate was in contact with cement, given the extent of osteophytes creating non-flat surfaces in some of the more degenerated specimens.
The other side of the vertebra was potted in a similar manner, however, due to the constraints of the potting container a measured quantity of cement was poured prior to lowering the vertebra into it.
A spirit level ensured parallel end-caps.

\subsection{Loading}

Following previous studies \cite{Wijayathunga2008}, the vertebrae were loaded with an initial maximum load of 800 N for similarly osteoporotic vertebrae.
However, after loading two of the initial set of vertebrae the stiffness continued to increase up to maximum 800 N. 
Following loads up to 2000 N showed that the stiffness reached a maximum between 1300 and 1500 N, after which the stiffness decreased indicating the start of damage to the vertebra.


\subsection{Vertebroplasty}

Despite the development of methods for the augmentation of bovine tail vertebrae, the methods for augmenting human vertebrae were altered due to the different geometry and density.
The human vertebrae, being much less dense, did not require the vertebroplasty needle to be inserted with the aid of a mallet.
Instead the needle could be pushed by hand through the cortical shell and into the vertebral body.

An additional difference was the approach with the needle, instead of entering the vertebrae through their pedicles an oblique approach was adopted.
This was due to variation in pedicle diameter between the L1 - L5 lumbar levels and therefore the potential to damage the region and its load sharing capabilities.
The oblique approach therefore avoided creating this damage to the pedicle-canal region, especially for the vertebrae with narrower pedicles and instead created much less damage to the vertebral body.

A final difference to the needle insertion methods was a change to the needle.
Here, a side opening needle was used, allowing the cement to be directed into the anterior-centre region of the vertebral body as opposed to directly out of the needle end.

Quantity?


\subsection{Modelling}

\section{Results}

\section{Discussion}

\section{Conclusion}







